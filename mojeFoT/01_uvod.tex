Respirační systém je jedním ze základních systémů lidského těla, který zajišťuje výměnů plynů mezi krví a okolním systémem, konkrétně úzce spolupracuje se srdcem a krví ve snaze extrahovat kyslík z vnějšího prostředí a zbavovat tělo nežádoucího oxidu uhličitého. \cite{muni} Stejně jako ostatní části lidského těla je třeba kontrolovat i respirační systém a jeho správnou funkci. K tomu slouží metody jako spirometrie nebo metoda nucených oscilací. 

Tento projekt je koncipován na základě jedné z~těchto diagnostických neinvazivních metod, kterou je metoda nucených oscilací realizovaná přístrojem Tremoflo C-100.  

Oproti spirometrii je metoda nucených oscilací přístupnější, protože vyžaduje pouze minimální spolupráci pacienta. Jako taková je vhodná jak pro děti \cite{StarczewskaDymek2019}, tak pro pacienty v oslabeném stavu u kterých by bylo obtížné provést měření, při kterém je třeba vyvíjet větší úsilí. \cite{Vlcek2018} Vyšetření probíhá v klidu, kdy pacient spontánně dýchá do přístroje přes jednorázový antibakteriální filtr. Vyhodnocuje se odpor dýchacích cest a tuhost plic. \cite{Vlcek2018}
Hlavní výhoda oproti klasické spirometrii je ta, že metoda nucených oscilací neměří plíce jako jeden globální systém, ale umožnuje lépe určit v~jakých místech plic se problematické místo nachází. \cite{Bhattarai27August2020}
Během měření přístroj generuje oscilace při různých frekvencích a amplitudách. Tyto oscilace jsou převedeny pomocí trubice nebo speciálního přívodu do~pacientova dýchacího systému. Následně se zaznamenává odezva respiračního systému. Výsledkem měření nucených oscilací je rezistence dýchacích cest, pružnost plic a další charakteristiky dýchání. Tato data mohou být následně využívána při diagnostice různých patologických stavů jako je např. CHOPN nebo astma. \cite{Vlcek2018}
Další možností využití FOT při diagnostice onemocnění plic, zejména vlivem kouření jsou popsány v~článku \cite{OliveiraRibeiro2018}.
Cílem této práce je zjistit závislost rezistence a reaktance modelu na jeho jednotlivých komponentech. \cite{Vlcek2018}
Model plic pro tuto metodu zatím neexistuje a dílčím cílem této práce je ho zhotovit.
