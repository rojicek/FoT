Respirační systém je jeden z nejdůležitějších orgánů v lidském těle. Vzhledem k jeho klíčové funkci je třeba ho průběžně monitorovat, zda provádí svoji funkci správně. Vzhledem k jeho významu je při nálezu choroby, jako tomu je u jiných nemocí, nutné akutně provést diagnózu. 
Metod pro diagnostiku respiračního systému je několik. Tento projekt je koncipován na základě jedné z diagnostických neinvazivních metod, kterou je metoda nucených oscilací realizované přístrojem tremoflo C-100. Cílem této práce je zjistit závislost jednotlivých komponent modelu na výsledných naměřených parametrech jako např.: rezistence a reaktance. \cite{Vlcek2018}


Aby se tato metoda dala využívat, je třeba znát výsledky jednotlivých parametrů jak pro zdravé plíce, tak pro různé patologické stavy se kterými se naměřené hodnoty mohou porovnat. Model plic pro tuto metodu zatím neexistuje a dílčím cílem této práce je ho zhotovit. 


Oproti spirometrii je metoda nucených oscilací přístupnější, protože vyžaduje pouze minimální spolupráci pacienta. Jako taková je vhodná jak pro děti, tak pro pacienty v oslabeném stavu u kterých by bylo obtížné provést měření, při kterém je třeba vyvíjet větší úsilí. \cite{Vlcek2018} Vyšetření probíhá v klidu, kdy pacient spontánně dýchá do přístroje přes jednorázový antibakteriální filtr. Vyhodnocuje se odpor dýchacích cest a tuhost plic. \cite{Vlcek2018}
Hlavní výhoda oproti klasické spirometrii je ta, že metoda nucených oscilací neměří plíce jako jeden globální systém, ale umožnuje lépe určit v jakých místech plic se problematické místo nachází. \cite{Vlcek2018}
Během měření přístroj generuje oscilace při různých frekvencích a amplitudách. Tyto oscilace jsou převedeny pomocí trubice nebo speciálního přívodu do pacientova dýchacího systému. Následně se zaznamenává odezva respiračního systému. Výsledkem měření nucených oscilací je rezistence dýchacích cest, pružnost plic a další charakteristiky dýchání. Tato data mohou být následně využívána při diagnostice různých patologických stavů jako je např. CHOPN nebo astma. \cite{Vlcek2018}
