Respirační systém je jedním ze základních systémů lidského těla, který zajišťuje výměnů plynů mezi krví a okolním systémem, konkrétně úzce spolupracuje se~srdcem a krví ve~snaze okysličovat lidské tělo a zbavovat ho nežádoucího oxidu uhličitého \cite{muni}. Stejně jako ostatní části lidského těla je třeba kontrolovat i respirační systém a jeho správnou funkci. Ke~kontrole správné mechanické činnosti plic slouží metody jako spirometrie nebo metoda nucených oscilací. 

Tento projekt se zabývá vývojem modelu pro~metodu nucených oscilací, což je diagnostická neinvazivní metoda, prováděna např. pomocí přístroje Tremoflo C-100.  


Spirometrie je diagnostická neinvazivní metoda pro~měření mechanické funkce plic. Její provedení vyžaduje dechové manévry ze~strany pacienta. Oproti spirometrii je metoda nucených oscilací přístupnější, protože vyžaduje pouze minimální spolupráci pacienta. Jako taková je vhodná jak pro děti \cite{StarczewskaDymek2019}, tak pro pacienty v oslabeném stavu u kterých by bylo obtížné provést měření, při kterém je třeba vyvíjet větší úsilí \cite{Vlcek2018}. Vyšetření probíhá v klidu, kdy pacient spontánně dýchá do přístroje přes jednorázový antibakteriální filtr. Vyhodnocuje se odpor dýchacích cest a tuhost plic \cite{Vlcek2018}.
Hlavní výhoda oproti klasické spirometrii je ta, že metoda nucených oscilací neměří plíce jako jeden globální systém, ale umožnuje lépe určit v~jakých místech plic se problematické místo nachází \cite{Bhattarai27August2020}.
Během měření přístroj generuje oscilace při různých frekvencích a amplitudách. Tyto oscilace jsou převedeny pomocí trubice nebo speciálního přívodu do~pacientova dýchacího systému. Následně se zaznamenává odezva respiračního systému. Výsledkem měření nucených oscilací je rezistence dýchacích cest, pružnost plic a další charakteristiky dýchání. Tato data mohou být následně využívána při diagnostice různých patologických stavů jako je např. CHOPN nebo astma \cite{Vlcek2018}.
Další možností využití FOT při diagnostice onemocnění plic, zejména vlivem kouření jsou popsány v~článku \cite{OliveiraRibeiro2018}.


Cílem této práce je sestrojit model respiračního systému. Ověření modelu bylo provedeno zjištěním závislosti rezistence a reaktance modelu na jeho jednotlivých komponentách.
Vedlejším cílem bylo zjistit jak jsme schopni ovlivnit výsledné naměřené parametry pomocí změn jednotlivých komponent. Funkčnost model, zejména opakovatelnost výsledků, byla ověřena s přístrojem Tremoflo.

