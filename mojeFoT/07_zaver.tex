V rámci tohoto projeku byl vytvořen model plic, který může sloužit jako fantom pro metodu nucených oscilací a je kompatibilní s přístrojem Tremoflo C-100. 

Model respiračního systému byl sestrojen pomocí skleněné nádoby, která představovala poddajnost, plastové trubice, která představovala internaci a parabolických rezistorů.  Schéma sestaveného modelu je vyobrazeno na obrázku \ref{obrazekschema}. 

Jeden z hlavních zdrojů byla bakalářská práce Bc. Tomáše Vlčka \cite{Vlcek2018}. Jeho práce byla zaměřena na srovnání spirometrie s metodou nucených oscilací. Má práce byla měřena na stejném přístroji, tudíž vychází ze stejných fyzikálních principů, proto je popis metod a přístrojů zejména v kapitole \ref{kap-metody} podobný. Hlavním rozdílem těchto dvou prací je, že já jsem sestavila vlastní model respiračního systému a měření prováděla na tomto modelu a nezkoumala jsem zdravotní stav lidí. Mým cílem bylo zjistit závislost změny jednotlivých komponent v modelu a naměřených parametrů rezistence a reaktance. Sestavení modelu na kterém bylo prováděno měření bylo součástí cíle této práce, který se podařilo splnit. 

Výsledky měření vykazují poměrně velké odchylky a pro přesnější výsledky by bylo třeba provést více měření a zjistit, co způsobuje pozorovanou nekonzistenci. Pro pokračování tohoto projektu je třeba také vyřešit problém s~exportem dat z~ovládacího software, aby se dalo pracovat i s~ostatními frekvencemi a ne pouze s frekvencí \SI{5}{Hz}, která je zobrazena jako výchozí na obrazovce po měření. Dále je třeba u každého měření provést více pokusů a prověřit zdali bude odchylka pořád stejně velká. 
Pokud se nepodaří měřit konzistentní data, tak by zřejmě nebylo možné v~tomto projektu pokračovat z důvodu nespolehlivosti přístroje na kterém bylo měření prováděno. 
