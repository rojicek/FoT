Projekt se zjišťuje závislost rezistance a reaktance a různých komponent v modelu respiračního systému. 
Cílem projektu bylo postavit model respiračního systému pomocí skleněné nádoby, která představovala poddajnost, plastové trubice, která představovala internaci a parabolických rezistorů.  Schéma sestaveného modelu je vyobrazeno na obrázku \ref{obrazekschema}. 

Jeden z hlavních zdrojů byla bakalářská práce Bc. Tomáše Vlčka. Jeho práce byla zaměřena na srovnání spirometrie s metodou nucených oscilací. Má práce byla měřena na stejném přístroji, tudíž vychází ze stejných fyzikálních principů a popis metod a přístrojů je podobný. Hlavním rozdílem těchto dvou pracích je, že já jsem pracovala s modelem respiračního systému a ne z lidmi. Cílem bylo zjistit závislost změny jednotlivých komponent v modelu a naměřených parametrů rezistence a reaktance. Sestavení modelu na kterém bylo prováděno měření bylo součástí cíle této práce, který se podařilo splnit. 

Výsledky měření vykazují poměrně velké odchylky a pro přesnější výsledky by bylo třeba provést více měření. Pro pokračování tohoto projektu je třeba vyřešit problém s exportem dat, aby se dalo pracovat i s ostatními frekvencemi a ne pouze s frekvencí \SI{5}{Hz}, která je zobrazena jako výchozí na obrazovce po měření. Dále je třeba u každého měření provést více pokusů a prověřit zdali bude odchylka pořád stejně velká. Kdyby se odchylka nezmenšovala, nebo naopak ještě zvětšovala musel by být projekt odkloněn jiným směrem z důvodu nespolehlivosti přístroje na kterém bylo měření prováděno. 
