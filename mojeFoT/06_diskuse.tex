
Hlavním cílem práce bylo zjistit závislost zvolných komponent v~modelu respiračního systému na~výsledných naměřených parametrech.  Celkem bylo provedeno 20~různých kombinací 3~velikostí parabolických rezistorů, 3~délek plastových trubic a 2~velikosti skleněných nádob. Měření jsem realizovala pomocí přístroje tremoflo C-100 (Thorasys Thoracic Medical Systems Inc., Kanada) v laboratoři FBMI ČVUT v Praze. 

Každá kombinace byla změřena 3x, aby se předešlo nežádoucím nepřesnostem měření. Přestože měření bylo prováděno vícekrát, tak hodnoty pokaždé vycházely s poměrně velkou odchylkou. Největší odchylka vznikala u~měření objemu. 
Do~systému byl v~pravidelných intervalech odpovídajících lidskému dechu vháněn \SI{1}{L} vzduchu, avšak čím byl v~modelu větší rezistor,  tím menší objem byl změřen. U Rp~5 se hodnota objemu pohybovala okolo  \SI{700}{ml}, u Rp~20 kolem  \SI{500}{ml} a u Rp~50 byl změřen objem okolo pouhých \SI{200}{ml}.

Přístroj měří na 8~různých frekvencích, jak bylo již zmíněné v kapitole \ref{kap-metody}. Následně všechny výsledky automaticky uloží do~tabulky. 
Bohužel celá tabulka všech frekvencí nešla vyexportovat, tudíž jsem pracovala pouze s výchozí frekvencí  \SI{5}{Hz}. 



