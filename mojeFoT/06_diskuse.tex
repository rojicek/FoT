
Hlavním výsledkem práce bylo sestrojení modelu respiračního systému a následně ho ověřit. Celkem bylo provedeno 18~různých kombinací: 3~velikostí parabolických rezistorů, 3~délek plastových trubic a 2~velikosti skleněných nádob. Měření jsem realizovala pomocí přístroje Tremoflo C-100 (Thorasys Thoracic Medical Systems Inc., Kanada) v laboratoři FBMI ČVUT v Praze. 


Do~systému byl v~pravidelných intervalech odpovídajících lidskému dechu vháněn \SI{1}{L} vzduchu, avšak čím byl v~modelu větší rezistor, tím menší objem byl změřen. U~Rp~5 se hodnota objemu pohybovala okolo  \SI{700}{ml}, u~Rp~20 kolem  \SI{500}{ml} a u~Rp~50 byl změřen objem okolo pouhých \SI{200}{ml}. V~ideálním případě by měl být u~všech odporů naměřen stejný objem \SI{1}{L}.

Přístroj měří na 8~různých frekvencích, jak bylo již zmíněno v~kapitole \ref{kap-metody}. Následně všechny výsledky automaticky uloží do~tabulky. 
Bohužel celá tabulka všech frekvencí nešla vyexportovat, tudíž jsem pracovala pouze s výchozí frekvencí  \SI{5}{Hz}. 

Každá kombinace byla změřena 3x, aby se předešlo nežádoucím nepřesnostem měření. Přestože měření bylo prováděno vícekrát, tak hodnoty pokaždé vycházely s poměrně velkou odchylkou. Největší odchylka vznikala u~měření objemu.  Vzhledem k počtu měření u jednotlivých kombinacích nelze provést věrohodné statické zpracování.
V~grafech v~kapitole~\ref{kap-vysledky} jsem záměrně zvolila stejné osy y pro rezistenci, resp. reaktanci. To umožnilo vizuálně srovnat opakovatelnost měření -  v~ideálním případě by v~každém grafu měly být hodnoty odpovídající stejné hodnotě nezávislé veličiny na ose x stejné, protože se jednalo o identická měření. Mnohde je tento předpoklad splněný, například graf \ref{rezistance-dily-20-odpor-20}, jinde jsou nekonzistence měření daleko větší, viz například \ref{rezistance_odpor_5_nadoba_54}.
U~rezistence je největší rozptyl kolem $\SI{1}{ cm\cdot H_{2}O}$ a u reaktace kolem $\SI{0,6}{ cm\cdot H_{2}O}$. 


Ještě méně uspokojivý je ale pozorovaný trend, očekávala jsem, že když například zafixuji velikost nádoby a nakreslím 3~grafy pro různé $R_p$, tak pozorovaný trend, v~tomhle případě závislost rezistance na délce trubice bude monotónní, tj vždy buď neklesající nebo nestoupající. To obvykle platí, ale některé série měření tohle nesplňují - viz třeba série měření
\ref{rezistance_dily_40_odpor_5}, \ref{rezistance_dily_40_odpor_20} a \ref{rezistance_dily_40_odpor_50} a to i když měření bylo relativně konzistentní.


