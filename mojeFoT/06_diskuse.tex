Hlavním cílem práce bylo zjistit závislost zvolných komponent v~modelu respiračního systému na~výsledných naměřených parametrech.  Celkem bylo provedeno 20~různých kombinací 3~velikostí parabolických rezistorů, 3~délek plastových trubic a 2~velikosti skleněných nádob. Měření jsem realizovala pomocí přístroje tremoflo C-100 (Thorasys Thoracic Medical Systems Inc., Kanada) v laboratoři FBMI ČVUT v Praze. 

Každá kombinace byla změřena 3x, aby se předešlo nežádoucím nepřesnostem měření. Přestože měření bylo prováděno vícekrát, tak hodnoty pokaždé vycházely s poměrně velkou odchylkou. Největší odchylka vznikala u~měření objemu. Do systému byl konstantně posílán objem 1~L vzduchu, avšak čím byl v~modelu větší rezistor,  tím meněí objem byl změřen. U Rp5 se hodnota objemu pohybovala okolo 700~ml, u Rp20 kolem 500~ml a u Rp50 byl změřen objem okolo pouhých 200~ml.

Přístroj měří na 8~různých frekvencích, jak bylo již zmíněné v kapitole \ref{kap-metody}. Následně všechny výsledky uloží do tabulky. Bohužel se nám celá tabulka všech frekvencí nepodařila vyexportovat, tudíž jsem pracovala pouze s výchozí frekvencí 5~Hz. 
