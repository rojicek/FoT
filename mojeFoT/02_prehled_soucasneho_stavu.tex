
Respirační systém je jedním ze základních systémů lidského těla, který zajišťuje výměnů plynů mezi krví a okolním systémem, konkrétně úzce spolupracuje se srdcem a krví ve snaze extrahovat kyslík z vnějšího prostředí a zbavovat tělo nežádoucího oxidu uhličitého. \cite{muni} Stejně jako ostatní části lidského těla je třeba kontrolovat i respirační systém a jeho správnou funkci. K tomu slouží metody jako spirometrie nebo metoda nucených oscilací. 

Spirometrie je diagnostická metoda pro měření ventilace respiračního systému. Je založena na měření tlaku na síťce uvnitř spirometru, který je úměrný objemu vzduchu. \cite{MEFANET} Základním fyzikálním principem je analogie Ohmova zákona, kdy ze~zná\-mé\-ho průtoku vzduchu, tj obdoba proudu a známé překážky, tj obdobě odporu je vypočtena změna tlaku, tj obdoba napětí. Spirometr zaznamenává výsledek jako graf ukazující objem plic v~závislosti na čase. \cite{Medicon} Spirometrie je určena pro měření statických a dynamických parametrů plic. Statický parametr  je velikost alveolárního prostoru, která informuje o případných restrikčních poruchách, příkladem je dechový objem, inspirační rezervní objem nebo vitální kapacita. Dynamický parametr je průběh proudění vzduchu v dýchacích cestách, který informuje o obstrukčních poruchách. Příkladem je časová vitální kapacita, maximální výdechový proud vzduchu nebo maximální volní ventilace. \cite{lekfak}
Spirometrie je v praxi velmi rozšířená diagnostická metoda i přes její základní nedostatek, kdy plíce jsou měřeny jako jeden globální systém a tudíž spirometrie není schopna rozlišit ve které části plic se nachází potenciální patologie.

Metoda nucených oscilací je novější diagnostická metoda měření ventilace respiračního systému. Funguje na podobném principu jako ostatní konvenční metody měření funkce respiračního systému, s tím rozdílem, že proud vzduchu v tomhle případě blokuje překážka ve formě pohyblivé síťky. Pohybem překážky vznikají tlakové rázy o frekvenci v řádu desítek hertzů. Pro měření metodou nucených oscilací (dále FOT - forced oscillation technique) není třeba žádné speciální dýchání ze strany pacienta. Přístroj měří klasickou spirometrii a zároveň vysílá pulzy s nízkou amplitudou a proměnlivou frekvencí do respiračního systému, a následně měří velikosti amplitud, které se vrátí zpátky do přístroje. \cite{Oostveen}

Každá frekvence má jiný dosah do jiné hloubky plic. Amplituda oscilací, která je naměřena po návratu do přístroje určuje inertance neboli setrvačnost plic v daném místě a následně tato informace pomáhá určit diagnózu. \cite{Oostveen}
Vynucené oscilace jsou superponované přímo na normální dýchání. Tato metoda vyšetření byla umožněna až s technologickým rozvojem počítačů. \cite{Vlcek2018} Během této doby bylo vyvinuto mnoho variant FOT s různými konfiguracemi měření, frekvencí oscilací a principy hodnocení. 

Přístroj tremoflo C-100 využívá novou metodu oscilometrie pro zjištění odporu  v dýchacích cestách a tuhosti plic bez speciálních dechových manévrů pacientů. Tato metoda patří mezi nejpokročilejší metody nucených oscilací. Tremoflo C-100 je zaměřen na měření plicních funkcí pomocí multifrekvenčních vln vysílaných do dechového oběhu pacienta. \cite{Nasinec}

Přístroj slouží k zjištění odporu v dýchacích cestách, posouzení tuhosti plic a rozlišení postižení centrálních a periferních dýchacích cest. Výsledky měření se zobrazují v softwaru vytvořeného přímo pro tento přístroj. \cite{Nasinec}

Pro vyhodnocení jsou známá data pro zdravého pacienta a dále jsou známá data pro vybrané konkrétní nemoci a jiné patologické stavy.  Po změření pacienta se jeho data srovnají s daty zdravého pacienta a podle odchylek v grafu se identifikují a analyzují potenciální patologie. 
