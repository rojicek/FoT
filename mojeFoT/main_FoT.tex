%% Šablona pro psaní závěrečných prací FBMI ČVUT
% Upraveno na základě požadavků na závěrečné práce schválených dne 24.2.2020.
%
% Vytvořeno ve spolupráci členů BRAIN Teamu.
% 
% Šablonu pro vás upravil/a:    Ing. Václava Piorecká, Ph.D.
% Kontakt:                      vaclava.piorecka@fbmi.cvut.cz

\documentclass[a4paper,12pt]{article}   % Definice - velikost dokumentu, základní velikost písma, typ
\usepackage[a4paper, top=2.5cm, left=3.5cm, right=2.5cm, bottom=2.5cm]{geometry}		% nastavení okrajů

%% Přidání balíčků podporujících různé funkcionality - ZÁKLADNÍ
\usepackage{amsmath,float}				% balíček pro matiku
\usepackage[dvipsnames]{xcolor}
\usepackage{color}
\usepackage{footnote}					% poznámka pod čarou
\usepackage{url}						% url odkazy
\DeclareUrlCommand\url{\def\UrlLeft{<}\def\UrlRight{>} \urlstyle{same}}   % nastavení URL odkazů pro podle normy ČSN ISO 690

\usepackage{float}						% plovoucí prostředí
\usepackage[utf8]{inputenc}			    % kódování	
\usepackage[czech]{babel}				% čeština
\usepackage{enumerate}      			% seznamy 
\usepackage{amsfonts}      				% množiny Z,R,N dvojitě
\usepackage{amssymb}      				% znaky úhlu a tak
\usepackage[pdftex]{graphicx}
\usepackage{setspace}
\usepackage{multicol}					% tabulka, slučování sloupců
\usepackage{multirow}                   % tabulka, slučování řádků
\usepackage{fancyhdr}					% záhlaví a zápatí stránky
\usepackage{chngcntr}                   % číslování (číslování rovnic, obrázků dle kapitol)
\usepackage{array}                      % rozšíření práce s tabulkami
\usepackage{helvet}                     % předefinuje \sfdefault to uhv (pro úvodní stránku)
\usepackage[flushleft]{threeparttable}  % prostředí pro tabulky - přidává vysvětlující poznámky pod tabulku
\usepackage{makecell}
\usepackage[title]{appendix}
\usepackage{siunitx}

\usepackage{caption}
\usepackage{subcaption}
\usepackage[section]{placeins}



\sisetup{output-decimal-marker = {,}}

% Následující balíčky řeší citace v dokumentu. 
\usepackage{csquotes}                       
\usepackage[style=iso-numeric]{biblatex}     
\addbibresource{citace.bib}             % zdrojový soubor s citacemi

%% Přidání balíčků podporujících různé funkcionality - VOLITELNÉ, DOPLŇKOVÉ
% Existuje celá řada dalších
\usepackage[]{algorithm2e}				    % balíček pro pseudokód
%\usepackage{ifthen}                        % pro algoritmy if else
%\usepackage{paralist}                      % Rozšířená možnost pro seznamy. Větší škála a možnosti jak seznamy dělat autoamticky. 
%\usepackage{fontspec}                      % specifické fonty
%\usepackage{icomma}      				    % není mezera za desetinnou čárkou
% \usepackage[titletoc,title]{appendix}     % automatické vytvoření příloh

%% Přidání speciálních příkazů 
\newcommand{\at}{\makeatletter @\makeatother}           % Vytiskne zavináč - \at
\newcommand{\degree}[1][]{\ensuremath{{#1}^\circ}}      % Vytiskne stupeň - \degree
\newcolumntype{C}[1]{>{\centering\let\newline\\\arraybackslash\hspace{0pt}}m{#1}}                                                     % zarovnání v tabulce, vycentrování, potřebuje \usepackage{array}


%% Ostatní definice a nastavení
\clubpenalty 10000		% penalizace sirotků, sirotek: poslední řádek odstavce je na nové stránce
\widowpenalty 10000		% penalizace vdov, vdova: první řádek nového odstavce je na konci stránky



\DeclareGraphicsExtensions{.pdf,.png,.jpg}	    % nahrávání obrázků, rošíření
\graphicspath{{obrazky/}} 				        % umístění obrázků

\counterwithin{figure}{section}         % číslování obrázků dle sekcí/kapitol
\counterwithin{table}{section}          % číslování tabulek dle sekcí/kapitol
\numberwithin{equation}{section}        % číslování rovnic dle sekcí/kapitol


\setlength{\parskip}{6pt}                   % odsazení mezi odstavci
\setlength{\parindent}{0.75cm}              % odsazení odstavců od okraje
\renewcommand{\baselinestretch}{1.20}	    % řádkování 1,2 - odpovídá pevnému řádkování 17 bodů

\usepackage{tocloft}                    % Nastaví tučné zvýraznění sekcí v obsahu
\renewcommand{\cftsecleader}{\cftdotfill{\cftdotsep}}   % přidá vodící linku do obsahu u sekcí


%% Ostatní neimplementované, pouze návod jak případně doplnit
% Vytvořit seznam použitých algoritmů a přejmenování názvu objektu na Algoritmus.
% \usepackage[algosection]{algorithm2e}
% \SetAlgorithmName{Algoritmus}{algorithm}{Seznam algoritmů}
   

%% Deklarace názvů - přepsat dle autora
\newcommand{\autor}{Adéla Rojíčková}
\newcommand{\vedouci}{Ing. Václav Ort, Ph.D.}
\newcommand{\nazev}{Model respiračního systému jako fantom pro~metodu nucených oscilací}
\newcommand{\nazevENG}{A~model of~the~respiratory system as a~phantom for~the~forced oscillation method}
\newcommand{\typ}{Semestrální projekt I}
\newcommand{\rok}{2023}
% Pro program a obor jsou instrukce zde: http://www.fbmi.cvut.cz/fakulta/uredni-deska 
% Nově akreditované mají pouze studijní programy
\newcommand{\program}{Biomedicínská a klinická technika}
\newcommand{\obor}{Biomedicínský technik}       % u nových akreditací vypustit

%% Vlastní začátek dokumentu
\begin{document}
    
	\pagestyle{empty}

	\begin{titlepage}
 		\begin{center}
 		\begin{figure}[!h]
			\centering
 			\includegraphics[width=0.2\textwidth]{symbol_cvut_konturova_verze}
 		\end{figure}
 		\textsf{\large{\textbf{ČESKÉ VYSOKÉ UČENÍ TECHNICKÉ V PRAZE}}}\\
 		%\vspace{0pt}   		
         {\color{NavyBlue}\makebox[\linewidth]{\rule{\textwidth}{0.4mm}}}
         %{\color{NavyBlue}\hrule }  	    \vspace{6pt}
 		\textsf{\normalsize{\textbf{FAKULTA BIOMEDICÍNSKÉHO INŽENÝRSVÍ}}}\\
		\textsf{\textbf{Katedra biomedicínské techniky}}\\	
		
		\vfill
 		
		\textsf{\Large{\textbf{\nazev}}}\\
	    \vspace{24pt}
		\textsf{\Large{\textbf{\nazevENG}}}\\
		\vspace{24pt}
		\textsf{\typ}\\ 
		\vfill
		\end{center}
		\textsf{Studijní program: \program} \\
		\textsf{Studijní obor: \obor} \\ % u nových akreditací zakomentovat či smazat tento řádek
		\\
		\textsf{Vedoucí práce: \vedouci}\\
				
		\begin{center}
		\textsf{\textbf{\autor}} \\ [0.5cm] 
		
		{\color{NavyBlue}\makebox[\linewidth]{\rule{\textwidth}{0.4mm}}} 
 		
		\textsf{\textbf{Kladno \rok}}
		\end{center}
			

		\clearpage


	\end{titlepage}

    
	\null\vfill
	\section*{Prohlášení}
	% \hspace ruší odsazení odstavce - v šabloně u prohlášení odstavce odsazené nejsou. 
    Prohlašuji, že jsem seminární práci s názvem \uv{\nazev} vypracovala samostatně a~použila k~tomu úplný výčet citací použitých pramenů, které uvádím v seznamu přiloženém k~seminární práci.

    \hspace{-0.75cm}Nemám závažný důvod proti užití tohoto školního díla ve smyslu \S 60 Zákona č.121/2000~Sb., o~právu autorském, o~právech souvisejících s právem autorským a~o~změně některých zákonů (autorský zákon), ve znění pozdějších předpisů. 
    
    \vspace{1em}
    
    \hspace{-0.75cm}V Kladně dne \today \hfill 

    \hspace{10cm} \textbf{\autor}

	\clearpage
	

	
	\null\vfill	
	\section*{ABSTRAKT}
        \subsection*{\nazev:}
	    Hlavním cílem této práce bylo zjistit závislost odezvy modelu plic na~jeho jednotlivých použitých komponentách. Model plic se skládal ze~skleněné nádoby, jejíž objem simuloval poddajnost plic, z~plastové trubice, která simulovala inertanci modelu a z~parabolického rezistoru. K~modelu byla připojena mechanická stříkačka, kterou se do~systému vháněl vzduch. Postupně byly použity 2 velikosti skleněných nádob, 3 délky plastových trubic a 3 velikosti parabolických rezistorů. Měření reaktance a rezistence bylo prováděno metodou nucených oscilací pomocí přístroje Tremoflo C-100 a probíhalo na~18 různých kombinacích modelu. Z~naměřených dat byla vyhodnocena reaktance a rezistence modelu v~závislosti na jeho aktuální konfiguraci.
	\subsection*{Klíčová slova}
		Akustická oscilometrie, metoda nucených oscilací, rezistence a reaktance modelu respiračního systému.
	\clearpage
		
		
	\null\vfill	
	\section*{ABSTRACT}
        \subsection*{\nazevENG:}
		 
        The main goal of this seminar paper was to find a~relation between the~lung model components and its response. The~lung model consisted of a glass receptacle whose capacity simulated the lungs compliance, plastic tube to simulate the inertness of said model, and parabolic resistor. To blow air into the system, two mechanical injectors were connected. Following components were used: 2 different sizes of the~receptacle, 3 plastic tubes of~different lengths and 3 different sizes of parabolic resistors. The~measurements of the~reactance and resistance were done using the~forced oscillation method with Tremoflo C-100. The~measurements have undergone different combinations of the~model. Obtained data of resistance and reactance were evaluated in relation according to the~actual configuration of the~model.
    
	\subsection*{Key words}
		Airwave oscillometry, forced oscillations technique, resistace and reactance of respiratory system model.
	\clearpage
	
    \pagestyle{plain}	% Číslování stránek začíná odsud
	
	\tableofcontents			% Vloží obsah
\clearpage

%\listoffigures
%\clearpage
%
%        \listoftables
%	\clearpage

	\section*{Seznam symbolů a zkratek} %sekce = nadpis, s * neni v obsahu
	\addcontentsline{toc}{section}{Seznam symbolů zkratek}
	\subsection*{Seznam symbolů}

\begin{table}[h]
	\label{tab:symboly}
	\catcode`\-=12          % Tento řádek je tam kvůli použití cline pro czech babel. Jinak to bere pomlčku jako znak a nevnímá ji jako rozsah. 
	\begin{center}
		\begin{tabular}{p{2.5cm}p{2.5cm}p{9.25cm}}
			\noalign{\hrule height 2pt}
			Symbol  & Jednotka              & Význam \\
			\noalign{\hrule height 2pt}

$Z_{rs} $ & $\SI{}{ cm\cdot H_{2}O \cdot s/L} $& Impedance \\
$R_{rs} $ & $\SI{}{ cm\cdot H_{2}O \cdot s/L} $& Rezistance \\
$X_{rs} $ & $\SI{}{ cm\cdot H_{2}O \cdot s/L} $& Reaktance \\
$f $ & $\SI{}{ Hz} $&  Oscilační frekvence  \\
$V $ & $\SI{}{ L} $& Objem \\
$Q $ & $\SI{}{ L/s} $& Průtok \\
$P $ & $\SI{}{ cm\cdot H_{2}O} $& Tlak \\

  

			\noalign{\hrule height 2pt}
	    \end{tabular}
	\end{center}
\end{table}

\subsection*{Seznam zkratek}
\begin{table}[h]
	\label{tab:zkratky}
	\catcode`\-=12          % Tento řádek je tam kvůli použití cline pro czech babel. Jinak to bere pomlčku jako znak a nevnímá ji jako rozsah. 
	\begin{center}
		\begin{tabular}{p{2.5cm}p{12.25cm}}
			\noalign{\hrule height 2pt}
			Zkratka  & Význam              \\
			\noalign{\hrule height 2pt}
			AOS& Akustická oscilometrie (Airwave oscillometry) \\
FOT  &Metoda nucených oscilací (Forced oscillation technique) \\
Z  &Impedance (Impedance) \\
R5 & Rezistace na frekvenci \SI{5}{Hz} (Rezistance at frequency of \SI{5}{Hz}) \\
X5  &Reaktance na frekvenci \SI{5}{Hz} (Reaktance at frequency of \SI{5}{Hz}) \\
			\noalign{\hrule height 2pt}
	    \end{tabular}
	\end{center}
\end{table}
	\clearpage
		
	    \addcontentsline{toc}{section}{Seznam tabulek}
	    \listoftables 		% seznam tabulek
	
	   \clearpage 			% kones stránky a odskok na další
		
	  \addcontentsline{toc}{section}{Seznam obrázků}
	  \listoffigures 		% seznam obrázků
	  \clearpage
	
		%\addcontentsline{toc}{section}{Seznam algoritmů}
		%\listofalgorithms
		%\clearpage
	
	
	\section{Úvod}
	Úvod obsahuje nejprve stručný obecný úvod do řešené problematiky (definuje oblast, kterou se práce zabývá, uvádí motivaci apod.). 
Obecný úvod má svým rozsahem tvořit velmi malou část celé práce.
	\clearpage
	
	\section{Přehled současného stavu}
	Oscilační metody byly poprvé použity v 60.~letech 20.~století.  \cite{Cap2000} Zpočátku byla tato metoda pouze monofrekvenční, později byl použit pravoúhlý elektrický signál obsahující všechny frekvence, dnes se používá sada vhodně zvolených frekvencí. Význam a růst využití této metody koreloval s~technickými pokroky ve vývoji výpočetní techniky. První zkušenosti s~touto metodou v České republice se datují ke konci 90.~let. 

\subsection{Spirometrie}
Spirometrie je diagnostická metoda pro měření ventilace respiračního systému. Je založena na měření tlaku na síťce uvnitř spirometru, který je úměrný objemu vzduchu. \cite{MEFANET} Základním fyzikálním principem je analogie Ohmova zákona, kdy ze~zná\-mé\-ho průtoku vzduchu, tj obdoba proudu a známé překážky, tj obdobě odporu je vypočtena změna tlaku, tj obdoba napětí. Spirometr zaznamenává výsledek jako graf ukazující objem plic v~závislosti na čase. \cite{Medicon} Spirometrie je určena pro měření statických a dynamických parametrů plic. Statický parametr  je velikost alveolárního prostoru, která informuje o případných restrikčních poruchách, příkladem je dechový objem, inspirační rezervní objem nebo vitální kapacita. Dynamický parametr je průběh proudění vzduchu v dýchacích cestách, který informuje o obstrukčních poruchách. Příkladem je časová vitální kapacita, maximální výdechový proud vzduchu nebo maximální volní ventilace. \cite{lekfak}
Spirometrie je v praxi velmi rozšířená diagnostická metoda i přes její základní nedostatek, kdy plíce jsou měřeny jako jeden globální systém a tudíž spirometrie není schopna rozlišit ve které části plic se nachází potenciální patologie.
\begin{figure}[!ht]
			\centering
 			\includegraphics[width=1\textwidth]{spirometrie-wiki}
			\caption{Spirometrické vyšetření \cite{spirowiki}}
			 \label{vysetreni}
 \end{figure}

\subsection{Metoda nucených oscilací}
Metoda nucených oscilací je novější diagnostická metoda měření ventilace respiračního systému. Funguje na podobném principu jako ostatní konvenční metody měření funkce respiračního systému, s tím rozdílem, že proud vzduchu v tomhle případě blokuje překážka ve formě pohyblivé síťky. \cite{Busschots2022} Pohybem překážky vznikají tlakové rázy o frekvenci v řádu desítek hertzů. Pro měření metodou nucených oscilací (dále FOT - forced oscillation technique) není třeba žádné speciální dýchání ze strany pacienta. Přístroj měří klasickou spirometrii a zároveň vysílá pulzy s nízkou amplitudou a proměnlivou frekvencí do respiračního systému, a následně měří velikosti amplitud, které se vrátí zpátky do přístroje. \cite{Oostveen}

Každá frekvence má jiný dosah do jiné hloubky plic. Amplituda oscilací, která je naměřena po návratu do přístroje určuje inertance neboli setrvačnost plic v daném místě a následně tato informace pomáhá určit diagnózu. \cite{Oostveen}
Vynucené oscilace jsou superponované přímo na normální dýchání. Tato metoda vyšetření byla umožněna až s technologickým rozvojem počítačů. \cite{Vlcek2018} Během této doby bylo vyvinuto mnoho variant FOT s různými konfiguracemi měření, frekvencí oscilací a principy hodnocení. 

\subsection{Přístroj Tremoflo C-100}
Přístroj Tremoflo C-100 využívá novou metodu oscilometrie pro zjištění odporu  v dýchacích cestách a tuhosti plic bez speciálních dechových manévrů pacientů. Tato metoda patří mezi nejpokročilejší metody nucených oscilací. Tremoflo C-100 je zaměřen na měření plicních funkcí pomocí multifrekvenčních vln vysílaných do dechového oběhu pacienta. \cite{Nasinec}

Přístroj slouží k zjištění odporu v dýchacích cestách, posouzení tuhosti plic a rozlišení postižení centrálních a periferních dýchacích cest. Výsledky měření se zobrazují v softwaru vytvořeného přímo pro tento přístroj. \cite{Nasinec}

Pro vyhodnocení jsou známá data pro zdravého pacienta a dále jsou známá data pro vybrané konkrétní nemoci a jiné patologické stavy.  Po změření pacienta se jeho data srovnají s daty zdravého pacienta a podle odchylek v grafu se identifikují a analyzují potenciální patologie. 

\begin{figure}[!ht]
			\centering
 			\includegraphics[width=0.7\textwidth]{tremoflo}
			\caption{Přístroj Tremoflo C-100 \cite{wwwpics}}
			 \label{tremoflopristroj}
 \end{figure}

Přístroj je kompaktní a přenosný. TremoFlo a jeho software jsou pro uživatele velmi přístupné a intuitivní. Měření je velmi rychlé (proběhne v rámci několika minut). UI softwaru nám poskytuje obraz měřených dat v reálném čase, přičemž jejich finální zpracování je vysoce detailní. Program též zaznamenává a zapisuje do své databáze výsledky měření jednotlivých pacientů.

Software pro analýzu dat z přístroje je velmi uživatelsky přístupný. 

\begin{figure}[!ht]
			\centering
 			\includegraphics[width=1\textwidth]{sw}
			\caption{Obrazovka ovládacího software \cite{wwwpics}}
			 \label{tremosw}
 \end{figure}

Další z~přístrojů využívající metodu nucených oscilací je například PulmoScan. Hlavní rozdíl přístrojů PulmoScan a Tremoflo C-100 je absence nutnosti drátového připojení. 
	\clearpage
	
      \section{Cíle práce}
	Cílem této práce je zjistit vliv změny jednotlivých komponent modelu na měřené respirační parametry, zejména rezistenci a reaktanci. 
K dosažení tohoto cíle bylo potřeba sestavit model části plic, konkrétně jsem sestavila model s jedním kompartmentem, tj. model jednoho laloku plic. Tento model bude sloužit jako fantom pro měření respiračních parametrů a bude kompatibilní s~přístrojem Tremoflo C-100.

	\clearpage
	
	\section{Metody}
	 \label{kap-metody}
\subsection {Akustická oscilometrie}
Invazivní metody měření respiračního systému se v současné době nevyužívají. Častěji se využívají metody neinvazivní jako např. spirometrie nebo akustická oscilometrie. Spirometrie je jedna z nejčastěji využívaných metod pro analýzu dýchacího systému, avšak k jejímu provedení je třeba spolupráce pacienta, který musí provádět hluboké nádechy a výdechy.
Akustická oscilometrie (dále AOS) má oproti spirometrii výhodu v tom, že vyžaduje pouze minimální spolupráci pacienta ve smyslu klidného spontánního dýchání. 
Je založená na základě měření impedance dýchacích cest. Výsledkem měření je kombinace hodnot rezistance a reaktance. Souhrnně se tyto dvě hodnoty nazývají impedancí. 

AOS je měřena pomocí přístroje tremoflo C-100. Podstatou funkce tohoto přístroje je akustické vlnění, které je vytvářeno pohybem síta. Akustické vlny jsou odporem dýchacích ces posunuty a deformovány a takto vzniklá oscilační akustická vlna je snímána senzory a počítačově zpracována
 Všechny zaznamenané hodnoty zpracuje software tremoFlo, jenž nakonec vypočítá veličinu impedance respiračního systému $Z_{rs}$. 

\begin{equation}
 \label{rce:1}
  \frac{P(f)}{Q(f)} = Z_{rs}(f) = R_{rs}(f) + j X_{rs}(f) 
\end{equation}
Kde P je tlak, Q je průtok a f je oscilační frekvence. 
Reálná část je označována jako rezistance $R_{rs}$, imaginární část je reaktance $X_{rs}$ a $j = \sqrt{-1}$. 
$R_{rs}$ představuje odpor vůči proudění vzduchu v plicích neboli, kolik tlaku je nutné pro průtok vzduchu dýchacími cestami. $X_{rs}$ znázorňuje při nízkých frekvencích tuhost tkání dýchacích cest. 


Technika nucených oscilací vysílá oscilace o velikosti přibližně \SI{2}{cm} sloupce $H_{2}O$, které se vytvoří v přístroji pomocí reproduktoru a následně se šíří do respiračního systému člověka. Reproduktor vytváří oscilační tlakové vlny na různých frekvencích. Nízké frekvence se šíří hluboko do plic, odkud se následně odrážejí zpátky do přístroje a vyšší frekvence se nedostanou hlouběji do plic, protože se  odrážejí zpátky do přístroje hned z periferních cest dýchacích. Tato skutečnost je daná fyzikálními vlastnostmi lidského těla, především velikostí a tvarem tkáňového složení lidského hrudníku. \cite{Vlcek2018} Frekvencí se k~měření používá devět (5, 11, 13, 17, 19, 23, 31 a  \SI{37}{Hz}). 

V přístroji je umístěn snímač tlaku a průtoku a ty přeměřují inspirační a expirační tlak plic a průtok dýchacích cest. 

Respirační impedance je součet rezistance a reaktance a je vypočítán z poměru tlaku  $P$ ku průtoku $Q$ u každé oscilační frekvence $f$. \cite{Vlcek2018}

\begin{equation}
	\label{rce:2}
	Z_{rs}(f) = \frac{P(f)}{Q(f)}
\end{equation}


V tomto projektu byl použit přístroj tremoflo C-100, který měří impedanci na frekvencích 5-37~Hz. Reaktance a rezistance jsou označovány $X$ a $R$ a v dolním indexu se nachází velikost frekvence na které byly měřeny, tj. např. pro frekvenci 5~Hz bude vypadat označení reaktance $X_5$ a rezistence $R_5$. 

Rezistance ($R_{rs}$) je veličina, která určuje centrální a periferní velikost odporu dýchacích cest. Velikost odporu dýchacích cest je zapříčiněna průchodností tlakové vlny vygenerované zařízením. Základní pevná frekvence pro oscilující tlaky je 5~Hz. Další frekvence se odvozují od tohoto základu odvozují. Do odvozených skupin frekvencí patří nízkofrekvenční signály (5-17~Hz), které se dostávají do obvodu centra plic, a vysokofrekvenční signály (19-37~Hz), jež pronikají pouze do proximálních dýchacích cest. 

Reaktance je imaginární část impedance. Jedná se o měřítko tuhosti plic, obzvlášť při nižších frekvencích. Toto měření vyplývá z pohybu vzduchu  a zpětné elasticitě plicní tkáně. Vzhledem k elastickým vlastnostem se tedy plíce při nízkých frekvencích pasivně rozšiřují a dochází tak k malému zpětnému rázu. Se zvyšující se energii dochází k přechodu plic z pasivního roztažení na aktivní. Čím vyšší je frekvence, tím víc energie putuje do plicního systému. Reakce na postupné zvyšování obsahu plic lze přirovnat k nafukování balónu - při jeho nafukování dojdeme k bodu, kdy při příjmu další energie začne balón vytvářet odpor, přičemž další přísun energie by mohl způsobit v balónu odraz.


\subsection{Kalibrace zařízení tremoflo}\label{kalibrace}
Přístroj, kterým bylo měření prováděno se nazývá TremoFlo C-100 (Thorasys Thoracic Medical Systems Inc., Kanada). Zařízení vyžaduje kalibraci před každým použitím. Kalibrace se provádí pomocí kalibrační zátěže, která je součástí balení. Kalibrační zátěž je označena konkrétním kódem, který se vloží do systému, následně  se nasadí na přenosný díl a spustí se kalibrace. Pro spolehlivé a přesné měření musí být přesnost 
v rozmezí $10\%$ nebo $\SI{0,1}{cm \cdot H_{2}O \cdot s/L}$. Pokud je tato podmínka splněna může se provést měření. \cite{Vlcek2018}

\subsection{Sestavení modelu}
Respirační systém se skládá se z pravé a levé plíce a průdušnic. Model respiračního systému  byl sestrojen pomocí mechanických analogií, skleněné nádoby, plastové trubice a průtočného rezistoru. Byly použity dvě velikosti nádob \SI{35}{L} a \SI{54}{L}, tři délky plastové trubice, \SI{20}{cm},  \SI{40}{cm},  \SI{60}{cm}  a tři různé parabolické rezistory PneuFlo Rp~5, Rp~20 a Rp~50 (Michigan Instruments, Michigan). 
Tremoflo C-100 je přístroj, který superponuje oscilace na spontánní dýchání člověka, tudíž model plic musí simulovat dýchání. Toto bylo vyřešeno mechanicky stříkačkou, která byla nastavena na  \SI{1}{L}, tudíž při každém stlačení vpustila do systému  \SI{1}{L} vzduchu. 
Postupně pomocí těchto součástek byly sestrojeny všechny kombinace respiračního systému a změřena odezva přístroje tremoflo C-100.

\subsection{Průběh měření}
Měření bylo prováděno v laboratoři pomocí přístroje tremoflo C-100 od firmy Thorasys. K měření byl třeba počítač s nainstalovaným softwarem pro tento přístroj. Na software tremoflo je třeba mít licenci, tudíž měření bylo možné provádět pouze na konkrétním počítači, kde je licence nainstalována.  Nejprve se přístroj i počítač zapojil do elektrické sítě a zapnul. Přístroj tremoflo c-100 se propojí s počítačem pomocí USB kabelu a po startu ovládacího software je potřeba provést kalibraci pomocí kalibrační zátěže, popis kalibrace je v podkapitole \ref{kalibrace}). Software nemá testovací režim, tudíž před měřením je třeba vytvořit kartu fiktivního pacienta. Do ní je třeba vyplnit  jméno, příjmení a věk pacienta. Po sestavení první kombinace modelu a vytvoření fiktivního pacienta se může přejít k měření. Každé měření probíhalo  \SI{16}{s} během kterých byla mechanicky stlačována střička, která do systému vháněla vzduch. Po  \SI{16}{s} přístroj data uložil a měření se opakovalo 3x kvůli snížení chyb a následnému průměrování. Po 3~měřeních jedné kombinace se jeden článek modelu, průtočný odpor, délka plastové trubice nebo velikost skleněné nádoby vyměnila a měření se opakovalo. Tímhle způsobem se vystřídaly všechny kombinace. Všechny data byla uložena v systému a potom se z~nich vygenerovala tabulka


	\clearpage
	
	\section{Výsledky}
	Měření probíhalo na 19 různých kombinacích modelu respiračního systému. Každé měření bylo prováděno 3x, aby se výsledek následně zprůměroval. 
Hodnoty u zdravé populace podle oficiálních stránek PulmoScan jsou uvedeny v tabulce XXXXXXXX.
Zde patří tabulka z \cite{PulmoScan}
	\clearpage
	
	\section{Diskuse}
	Hlavním cílem práce bylo zjistit závislost zvolných komponent v modelu respiračního systému na výsledných naměřených parametrech.  Celkem bylo provedeno 19 různých kombinací 3 velikostí průtočných odporů, délek plastových trubic a 2 velikosti skleněných nádob. Měření jsem realizovala pomocí přístroje tremoflo C-100 (Thorasys Thoracic Medical Systems Inc., Kanada) v laboratoři FBMI ČVUT v Praze.
	\clearpage
	
	\section{Závěr}
	V rámci tohoto projeku byl vytvořen model plic, který může sloužit jako fantom pro metodu nucených oscilací a je kompatibilní s přístrojem Tremoflo C-100. 

Model respiračního systému byl sestrojen pomocí skleněné nádoby, která představovala poddajnost, plastové trubice, která představovala internaci a parabolických rezistorů.  Schéma sestaveného modelu je vyobrazeno na obrázku \ref{obrazekschema}. 


Jeden z hlavních zdrojů byla bakalářská práce Bc. Tomáše Vlčka \cite{Vlcek2018}. Jeho práce byla zaměřena na srovnání spirometrie s metodou nucených oscilací. Má práce byla měřena na stejném přístroji, tudíž vychází ze stejných fyzikálních principů, proto je popis metod a přístrojů zejména v kapitole \ref{kap-metody} podobný. Hlavním rozdílem těchto dvou prací je, že já jsem sestavila vlastní model respiračního systému a měření prováděla na tomto modelu a nezkoumala jsem zdravotní stav lidí. 


Výsledkem projektu bylo sestrojit model respiračního systému a následně zjistit závislost změny jednotlivých komponent v modelu a naměřených parametrů rezistence a reaktance. Se~zvětšujícím se odporem dochází k úniku vzduchu docházejícího do~přístroje. Při měření s~menší nádobou jsou výsledky konzistentnější než při~měření s~větší nádobou. Při~využití Rp 20 je rezistence a reaktance o~cca~$\SI{0,5}{cm \cdot H_{2}O \cdot s/L}$ vyšší než při~použití Rp 5 a Rp 50. 


Výsledky měření vykazují poměrně velké odchylky a pro přesnější výsledky by bylo třeba provést více měření a zjistit, co způsobuje pozorovanou nekonzistenci. Pro pokračování tohoto projektu je třeba také vyřešit problém s~exportem dat z~ovládacího software, aby se dalo pracovat i s~ostatními frekvencemi a ne pouze s frekvencí \SI{5}{Hz}, která je zobrazena jako výchozí na obrazovce po měření. Dále je třeba u každého měření provést více pokusů a prověřit zdali bude odchylka pořád stejně velká. 
Pokud se nepodaří měřit konzistentní data, tak by zřejmě nebylo možné v~tomto projektu pokračovat z důvodu nespolehlivosti přístroje na kterém bylo měření prováděno. 

	\clearpage

%-------------Literatura-------------------
    \clearpage	
    \renewcommand{\refname}{Seznam použité literatury} 	% Přejmenování Reference
    \addcontentsline{toc}{section}{Seznam použité literatury}     % Přidání této kapitoly do obsahu
    \printbibliography
    \clearpage

%-------------Přílohy----------------------
  \begin{appendices}
\section{Výsledky měření}
\midinsert \clabel[tvar1]{Tabulka varianta 1}
\ctable{c|r|r|r|r|r|r}
{
 \hfil         &	$R_{5}$    &	$X_{5}$  & $F_{res}$ &	Reference &	$V_{T}$ & $COH_{5}$ \cr
 \hfil  $Měření$ & $[cm \cdot H_{2}O \cdot s / L]$     &	  $[cm \cdot H_{2}O \cdot s / L]$  &	$[Hz]$  & 	 & [L] 	 &        \crl \tskip4pt
1 &	3,16 &	-0,02 &	5,43 &	18,69 &	0,58 &	0,98 \cr
2 &	2,64 &	0,41 &	n/a &	18,69 &	0,73 &	0,97 \cr
3 &	2,34 &	0,35 &	n/a &	18,69 &	0,73 &	0,98 \cr
}
\caption/t Díly: 3, odpor: 20, nádoba: 54	
\endinsert


\midinsert \clabel[tvar2]{Tabulka varianta 2}
\ctable{c|r|r|r|r|r|r}
{
 \hfil         &	$R_{5}$    &	$X_{5}$  & $F_{res}$ &	Reference &	$V_{T}$ & $COH_{5}$ \cr
 \hfil  $Měření$ & $[cm \cdot H_{2}O \cdot s / L]$     &	  $[cm \cdot H_{2}O \cdot s / L]$  &	$[Hz]$  & 	 & [L] 	 &        \crl \tskip4pt
1 &	2,37 &	0,31 &	n/a	 &18,69	 &0,72	 &0,98\cr
2 &	3,1	& 0,01 &	n/a	 &18,69	 &0,51 &	0,99\cr
3 &	3,1	& 0,08 &	n/a &	18,69 &	0,47 &	0,99\cr
}
\caption/t Díly: 3, odpor: 5, nádoba: 54	
\endinsert


\midinsert \clabel[tvar3]{Tabulka varianta 3}
\ctable{c|r|r|r|r|r|r}
{
 \hfil         &	$R_{5}$    &	$X_{5}$  & $F_{res}$ &	Reference &	$V_{T}$ & $COH_{5}$ \cr
 \hfil  $Měření$ & $[cm \cdot H_{2}O \cdot s / L]$     &	  $[cm \cdot H_{2}O \cdot s / L]$  &	$[Hz]$  & 	 & [L] 	 &        \crl \tskip4pt
1&	2,47&	0,06&	n/a	&    18,69&	0,23&	1\cr
2&	2,44&	0,03&	n/a&	18,69&	0,22&	1\cr
3&	2,45&	0,06&	n/a&	18,69&	0,24&	1\cr
}
\caption/t Díly: 3, odpor: 50, nádoba: 54	
\endinsert


\midinsert \clabel[tvar4]{Tabulka varianta 4}
\ctable{c|r|r|r|r|r|r}
{
 \hfil         &	$R_{5}$    &	$X_{5}$  & $F_{res}$ &	Reference &	$V_{T}$ & $COH_{5}$ \cr
 \hfil  $Měření$ & $[cm \cdot H_{2}O \cdot s / L]$     &	  $[cm \cdot H_{2}O \cdot s / L]$  &	$[Hz]$  & 	 & [L] 	 &        \crl \tskip4pt
1&	2,47&	0,1	&n/a&	18,69&	0,27&	1\cr
2&	2,42&	0&	n/a	&18,69&	0,24&	1\cr
3&	2,44&	0,1&	n/a&	18,69&	0,22&	1\cr
}
\caption/t Díly: 2, odpor: 50, nádoba: 54	
\endinsert


\midinsert \clabel[tvar5]{Tabulka varianta 5}
\ctable{c|r|r|r|r|r|r}
{
 \hfil         &	$R_{5}$    &	$X_{5}$  & $F_{res}$ &	Reference &	$V_{T}$ & $COH_{5}$ \cr
 \hfil  $Měření$ & $[cm \cdot H_{2}O \cdot s / L]$     &	  $[cm \cdot H_{2}O \cdot s / L]$  &	$[Hz]$  & 	 & [L] 	 &        \crl \tskip4pt
1&	3,24&	0,04&	n/a	&18,69&	0,53&	0,99\cr
2&	3,15&	0,02&	n/a	&18,69&	0,57&	0,98\cr
3&	3,17&	0,06&	n/a&	18,69&	0,54&	0,99\cr
}
\caption/t Díly: 1, odpor: 20, nádoba: 54	
\endinsert



\midinsert \clabel[tvar6]{Tabulka varianta 6}
\ctable{c|r|r|r|r|r|r}
{
 \hfil         &	$R_{5}$    &	$X_{5}$  & $F_{res}$ &	Reference &	$V_{T}$ & $COH_{5}$ \cr
 \hfil  $Měření$ & $[cm \cdot H_{2}O \cdot s / L]$     &	  $[cm \cdot H_{2}O \cdot s / L]$  &	$[Hz]$  & 	 & [L] 	 &        \crl \tskip4pt
1&	2,27&	0,26&	n/a	&18,69&	0,74&	0,98\cr
2&	2,46&	0,16&	n/a&	18,69&	0,71&	0,98\cr
3&	2,77&	0,2&	n/a	&18,69&	0,71&	0,97\cr
}
\caption/t Díly: 1, odpor: 5, nádoba: 54	
\endinsert


\midinsert \clabel[tvar7]{Tabulka varianta 7}
\ctable{c|r|r|r|r|r|r}
{
 \hfil         &	$R_{5}$    &	$X_{5}$  & $F_{res}$ &	Reference &	$V_{T}$ & $COH_{5}$ \cr
 \hfil  $Měření$ & $[cm \cdot H_{2}O \cdot s / L]$     &	  $[cm \cdot H_{2}O \cdot s / L]$  &	$[Hz]$  & 	 & [L] 	 &        \crl \tskip4pt
1&	3,11&	-0,01&	12,18&	18,69&	0,51&	0,98\cr
2&	3,27&	-0,11&	7,93&	18,69&	0,45&	0,97\cr
3&	3,11&	-0,03&	6,57&	18,69&	0,52&	0,97\cr
}
\caption/t Díly: 2, odpor: 20, nádoba: 54	
\endinsert


\midinsert \clabel[tvar8]{Tabulka varianta 8}
\ctable{c|r|r|r|r|r|r}
{
 \hfil         &	$R_{5}$    &	$X_{5}$  & $F_{res}$ &	Reference &	$V_{T}$ & $COH_{5}$ \cr
 \hfil  $Měření$ & $[cm \cdot H_{2}O \cdot s / L]$     &	  $[cm \cdot H_{2}O \cdot s / L]$  &	$[Hz]$  & 	 & [L] 	 &        \crl \tskip4pt
1&	3,64&	0,01&	n/a	&18,69&	0,63&	0,98\cr
2&	3,75&	-0,16&	7,5&	18,69&	0,59&	0,97\cr
3&	3,59&	0,08&	n/a&	18,69&	0,58&	0,97\cr
}
\caption/t Díly: 2, odpor: 20, nádoba: 35	
\endinsert

\midinsert \clabel[tvar9]{Tabulka varianta 9}
\ctable{c|r|r|r|r|r|r}
{
 \hfil         &	$R_{5}$    &	$X_{5}$  & $F_{res}$ &	Reference &	$V_{T}$ & $COH_{5}$ \cr
 \hfil  $Měření$ & $[cm \cdot H_{2}O \cdot s / L]$     &	  $[cm \cdot H_{2}O \cdot s / L]$  &	$[Hz]$  & 	 & [L] 	 &        \crl \tskip4pt
1 &	2,94 &	0,04 &	n/a &	18,69 &	0,32 &	1\cr
2 &	2,93 &	0,03 &	n/a	 &18,69 &	0,31 &	1\cr
3&	2,88&	0,12&	n/a	&18,69&	0,27&	1	\cr
}
\caption/t Díly: 2, odpor: 50, nádoba: 35	
\endinsert

\midinsert \clabel[tvar10]{Tabulka varianta 10}
\ctable{c|r|r|r|r|r|r}
{
 \hfil         &	$R_{5}$    &	$X_{5}$  & $F_{res}$ &	Reference &	$V_{T}$ & $COH_{5}$ \cr
 \hfil  $Měření$ & $[cm \cdot H_{2}O \cdot s / L]$     &	  $[cm \cdot H_{2}O \cdot s / L]$  &	$[Hz]$  & 	 & [L] 	 &        \crl \tskip4pt
1&	2,19&	0,04&	n/a	& 18,69&	0,75&	0,97	\cr
2&	2,39&	-0,05&	5,54&	18,69&	0,76&	0,97	\cr
3&	2,25&	0&	n/a&	18,69&	0,75&	0,97\cr
}
\caption/t Díly: 2, odpor: 5, nádoba: 35	
\endinsert

\midinsert \clabel[tvar11]{Tabulka varianta 11}
\ctable{c|r|r|r|r|r|r}
{
 \hfil         &	$R_{5}$    &	$X_{5}$  & $F_{res}$ &	Reference &	$V_{T}$ & $COH_{5}$ \cr
 \hfil  $Měření$ & $[cm \cdot H_{2}O \cdot s / L]$     &	  $[cm \cdot H_{2}O \cdot s / L]$  &	$[Hz]$  & 	 & [L] 	 &        \crl \tskip4pt
1&	3,81&	-0,18&	9,8	&18,69	&0,6	&0,97\cr
2&	3,96&	-0,17	&6,7	&18,69	&0,61&	0,97\cr
3&	3,65	&0,08&	n/a	&18,69&	0,64&	0,97\cr
}
\caption/t Díly: 2, odpor: 20, nádoba: 35	
\endinsert
\end{appendices}



    
\end{document}