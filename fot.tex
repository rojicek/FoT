

\input ctustyle2
\worktype [O/CZ]
\workname {Ročníková práce nebo jak se to jmenuje}
\faculty {F7}
\department {Katedra permoníků}
\title {Model respiračního systému jako fantom pro metodu nucených oscilací }
\author {Adéla Rojíčková}
\date {červen 2023}
\abstractEN {This document is to show how smart I am.}
\abstractCZ {Tento dokument je abych měla prázdniny.}
\declaration {Prohlašuji, že jsem se neflákala.}
\thanks{díkes, díkes}
\input opmac-bib

\makefront


\chap Přehled současného stavu

Respirační systém \cite[VJgRTvR9gAkAHd6X], \cite[aaa] je jedním ze základních systémů lidského těla, který zajišťuje výměnů plynů mezi krví a okolním systémem, konkrétně úzce spolupracuje se srdcem a krví ve snaze extrahovat kyslík z vnějšího prostředí a zbavovat tělo nežádoucího oxidu uhličitého. [2] Stejně jako ostatní části lidského těla je třeba kontrolovat i respirační systém a jeho správnou funkci. K tomu slouží metody jako spirometrie nebo metoda řízených oscilací. 

Spirometrie je diagnostická metoda pro měření ventilace respiračního systému. Je založena na měření tlaku na síťce uvnitř spirometru, který je úměrný objemu vzduchu. Základním fyzikálním principem je analogie Ohmova zákona, kdy ze známého průtoku vzduchu, tj obdoba proudu a známé překážky, tj obdobě odporu je vypočtena změna tlaku, tj obdoba napětí. Spirometr zaznamenává výsledek jako graf ukazující objem plic v závislosti na čase. [1] Spirometrie je určena pro měření statických a dynamických parametrů plic. Statický parametr  je velikost alveolárního prostoru, která informuje o případných restrikčních poruchách, příkladem je dechový objem, inspirační rezervní objem nebo vitální kapacita. Dynamický parametr je průběh proudění vzduchu v dýchacích cestách, který informuje o obstrukčních poruchách. Příkladem je časová vitální kapacita, maximální výdechový proud vzduchu nebo maximální volní ventilace. [3]
Spirometrie je v praxi velmi rozšířená diagnostická metoda i přes její základní nedostatek, kdy plíce jsou měřeny jako jeden globální systém a tudíž spirometrie není schopna rozlišit ve které části plic se nachází potenciální patologie.

Metoda nucených oscilací je novější diagnostická metoda měření ventilace respiračního systému. Funguje na podobném principu jako ostatní konvenční metody měření funkce respiračního systému, s tím rozdílem, že proud vzduchu v tomhle případě blokuje překážka ve formě pohyblivé síťky. Pohybem překážky vznikají tlakové rázy o frekvenci v řádu desítek hertzů. Pro měření metodou FOT (forced oscillation technique) není třeba žádné speciální dýchání ze strany pacienta. Přístroj měří klasickou spirometrii a zároveň vysílá pulzy s nízkou amplitudou a proměnlivou frekvencí do respiračního systému a následně měří velikosti amplitud, které se vrátí zpátky do přístroje. [7]

Každá frekvence má jiný dosah do jiné hloubky plic, podle toho jaká amplituda oscilací se vrátí do přístroje, tak určí jaká je inertance neboli setrvačnost plic v daném místě. 

Metoda řízených oscilací je schopna změřit poddajnost a inertanci v různých částech respiračního systému a následně diagnostikovat určité plicní patologie. 
Výhodou tohoto vyšetření je jeho neinvazivnost a nenáročná spolupráce pacienta, není třeba provádět nějaké speciální dýchání. Vynucené oscilace jsou superponovány přímo na normální dýchání. 

Tato metoda vyšetření se stává populární během posledních deseti let. Během té doby bylo vyvinuto mnoho variant FOT s různými konfiguracemi měření, frekvencí oscilací a principy hodnocení. 

Přístroj TremoFlo C-100 využívá novou metodu oscilometrie pro zjištění odporu  v dýchacích cestách a tuhosti plic bez speciálního dýchání pacientů. Tato metoda patří mezi nejpokročilejší metody nucených oscilací. TremoFlo C-100 je zaměřen na měření plicních funkcí pomocí multifrekvenčních vln vysílaných do dechového oběhu pacienta. [4]

Přístroj slouží k zjištění odporu v dýchacích cestách, posouzení tuhosti plic a rozlišení postižení centrálních a periferních dýchacích cest. Je vhodný i pro nespolupracující pacienty a ty, pro které je provedení vyšetření klasické spirometrie náročné. 
Výsledky měření se zobrazují v softwaru vytvořeného přímo pro tento přístroj. [4]

Pro vyhodnocení jsou známá data pro zdravého pacienta a dále jsou známá data pro vybrané konkrétní nemoci a jiné patologické stavy.  Po změření pacienta se jeho data srovnají s daty zdravého pacienta a podle odchylek v grafu se identifikují a analyzují potenciální patologie. 

Cílem projektu je vytvořit model respiračního systému pro aplikaci metody nucených oscilací. 

\bibchap
\usebib/c (simple) fotbib

\bye
