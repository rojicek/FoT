Kapitola obsahuje detailní popis způsobu řešení problému studentem. 
V závislosti na charakteru řešeného problému je tuto část práce možné rozdělit do více kapitol/podkapitol, kdy názvy kapitol jsou voleny konkrétněji s ohledem na jejich obsah.

Popisovány jsou postupy aplikované k dosažení výsledků práce a rovněž např. použité přístroje a materiál, metody zpracování dat a jejich statistického vyhodnocení apod. 
V případě měření s živými subjekty tato část práce obsahuje informaci, jak byly ošetřeny etické otázky výzkumu a charakteristiku subjektů dle zvyklostí v biomedicínských časopisech.
Struktura a obsah této části je detailně probírána a~procvičována v příslušných seminářích na oboru BMT.
V případě, že text obsahuje matematický vzorec, na který se bude text později odkazovat, uvádějte vzorec na samostatném řádku, vycentrovaný na střed řádku a s číslem, které udává pořadí mezi číslovanými vzorci v kapitole, jako je tomu v příkladu vztahu pro elektrický odpor $R$:
\begin{equation}
	\label{rce:odpor}
	R = \frac{U}{I}
\end{equation}
kde $U$ je napětí a $I$ je proud. 
Pokud je vzorec součástí věty, jako v předchozím vztahu~\ref{rce:odpor}, pokračujte za ním textem bez odsazení nového odstavce. 
Vzorce vkládejte pomocí možností editoru, nekopírujte vzorce z jiných pramenů.  

\subsection{Ukázka citací}
Ukázka citací:
Byl využit software Matlab \cite{MATLAB} s doplňkovým toolboxem EEGLAB~\cite{eeglab}. 
Princip algoritmu k-means najdete například v publikaci~\cite{Krajca2011}.

Ukázka více citací~\cite{MATLAB, eeglab}.