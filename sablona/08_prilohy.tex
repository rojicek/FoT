\section*{Příloha A: Požadavky na formátování práce}
    \label{app:pozadavky}
    \addcontentsline{toc}{section}{Příloha A: Požadavky na formátování práce}     % Přidání této kapitoly do obsahu

    \begin{itemize}
        \item Pro hlavní text práce používejte patkové písmo (Times New Roman, Georgia, Garamond apod.), velikost 12. Rovnice, matematické symboly apod. by měly být sázeny stejným, nebo alespoň co nejpodobnějším písmem stejné velikosti. Popisy obrázků a tabulek sázejte stejným písmem se zmenšenou velikostí. Nadpisy, čísla stránek, případné záhlaví či zápatí apod. mohou být sázeny buď stejným písmem jako hlavní text, nebo písmem bezpatkovým (např. Calibri). \textbf{V celé práci musí být použity maximálně dvě různá písma.}
        \item Řádkování práce, odsazení odstavců, velikosti písma v nadpisech apod. definují přímo jednotlivé styly použité v této šabloně.
        \item Okraje stránek práce jsou vždy 2,5 cm na každé straně plus 1 cm u hřbetu práce (levá strana).
        \item Práce je tištěna jednostranně, na papír formátu A4.
        \item Stránky se číslují arabskými číslicemi počínaje první (titulní) stranou. Číslování stránek se zobrazuje až od první stránky obsahu, což znamená, že na titulní straně, v zadání, prohlášení, poděkování a abstraktech se číslo stránky neuvádí.
        \item Hlavní kapitoly práce, počítaje v to Úvod a Závěr, jsou číslovány arabskými číslicemi. Seznam použité literatury číslo nemá. Přílohy označujte velkými písmeny anglické abecedy.
        \item Každou hlavní kapitolu práce (nadpis 1. úrovně) začínejte na samostatné stránce.
    \end{itemize}
\clearpage

\section*{Příloha B: Základní typografické zásady}
    \label{app:typo}
    \addcontentsline{toc}{section}{Příloha B: Základní typografické zásady}     % Přidání této kapitoly do obsahu
    
    \begin{itemize}
        \item Fyzikální a fyziologické veličiny a matematické proměnné se sázejí proloženě (kurzívou). Zkratky a symboly, pod kterými se neskrývá číselná hodnota, jsou sázeny normálním písmem – stejně jako označení fyzikálních jednotek.
        \item Jednotky veličin a symboly (například procenta) se v textu od číselných údajů oddělují nezlomitelnou mezerou. Zápis bez mezery má význam přídavného jména. Např. $10\,\Omega$ čteme \uv{deset ohmů} a $10\Omega$ čteme \uv{desetiohmový}.
        \item Nezlomitelnou mezeru je nutné v editoru textu vyznačit. Např. v aplikaci Microsoft Word se použije kombinace Shift, Ctrl, mezerník. 
        \item Neslabičné předložky a spojky (netýká se \uv{a}) nesmí zůstat na konci řádku. Proto za nimi používejte nezlomitelnou mezeru.
        \item Rozlišujte spojovník a pomlčku. Spojovník je krátká čára používaná ke spojení dvou slov (např. česko-anglický slovník). Pomlčka slouží k vyznačení prodlevy v textu, pak ji obvykle píšeme s mezerami, nebo k vyznačení rozsahu (5–10), kdy se píše bez mezer.
        \item Pro podrobnější informace k typografii doporučujeme např. dokument prof. Roubíka Fyzikální veličiny a číselné údaje, dostupný na stránce:
\url{https://predmety.fbmi.cvut.cz/cs/17PMBPIZ}
a dokument Jany Borůvkové Jak napsat bakalářskou práci, dostupný na stránce:
\url{http://is.mendelu.cz/dok_server/slozka.pl?id=53294;download=160152;lang=cz}
        \item Pro zajištění jazykové správnosti práce doporučujeme konzultovat Internetovou jazykovou příručku Ústavu pro jazyk český Akademie věd ČR dostupnou z: 
\url{http://prirucka.ujc.cas.cz/}
    \end{itemize}
    
\clearpage

\section*{Příloha C: Další doporučení pro přehlednost textu}
    \label{app:doporuceni}
    \addcontentsline{toc}{section}{Příloha C: Další doporučení pro přehlednost textu}     % Přidání této kapitoly do obsahu
    
    \begin{itemize}
        \item Obrázky a tabulky sázejte v textu samostatně, bez obtékání textu po stranách. Nevkládejte obrázky a tabulky na stránku před skončením odstavce. Zkontrolujte, že popis obrázku nebo tabulky zůstal na stejné straně jako vlastní obrázek nebo tabulka.
        \item První řádek odstavce by neměl zůstat sám na konci řádky (tzv. vdova) a poslední řádek odstavce by neměl zůstat sám na začátku nové stránky (tzv. sirotek).
        \item Veškeré zkratky, s výjimkou těch nejznámějších jako DNA, by měly být v práci vysvětleny při prvním výskytu v hlavním textu a současně také v abstraktu, pokud je nutné je v něm použít.
        \item Na rovnice odkazujte jejich číslem, a to až za jejich uvedením v textu práce.
        \item Všechny obrázky a tabulky v práci musí být odkazovány z hlavního textu pomocí svých čísel. 
    \end{itemize}

\clearpage

\section*{Příloha D: Obsah přiloženého CD/DVD}
    \label{app:obsah}
    \addcontentsline{toc}{section}{Příloha D: Obsah přiloženého CD/DVD}     % Přidání této kapitoly do obsahu
    
    Poslední přílohou práce je obsah přiloženého datového nosiče. 
    Typ a povinný obsah datového nosiče je specifikován na stránkách FBMI ČVUT v Praze (\url{https://www.fbmi.cvut.cz/cs/student/bakalarske-diplomove-prace}).
    
    Dále na datový nosič umístěte přílohy, které není možné pro jejich rozsah nebo charakter umístit do výtisku práce, ale které mohou být důležité pro posouzení úplnosti a kvality splnění zadání práce, jako jsou různé konstrukční výkresy, zdrojový kód programů pro zpracování naměřených dat apod.